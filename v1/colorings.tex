\section{Colorings}\label{sec:colorings}

\begin{itemize}
    \item A graph has been ($n$-)\textbf{colored} if each vertex has been assigned a number from $\{1, 2, \ldots, n\}$ such that no edge joins vertices with the same number (``color'').  We say that a graph $G$ is \textbf{$n$-colorable} if it may be $n$-colored.
    \item examples
    \item The \textbf{chromatic number} of a graph $G$ is the smallest $n$ such that $G$ is $n$-colorable, denoted $X(G)$.
    \item examples
    \item (DNP) Four-color theorem: Every planar graph has $X \leq 4$.
    \item Five-color theorem: Every planar graph has $X \leq 5$. (induction)
    \item Every planar graph having a vertex of degree $\leq 4$ has $X\leq 4$.  This is crazy!  Compare to the theorem about every planar graph having a vertex $\leq 5$. % Wait, I don't believe this as stated
    \item It is sufficient to prove the four-color theorem for cubic maps (trivalent graphs).
    \item reading about the four-color theorem and its proof.  Do you believe it?
    \item Map colorings!  define dual graph and why no bridges
    \item exercises
    \item (Brooks) Let $G$ be a connected (simple) graph.  If $G$ is not complete and its largest vertex degree is $n$, then $G$ is $n$-colorable. % I show them? tricky proof!
    \item discuss examples where Brooks theorem is useful vs. not
    \item A map is 2-colorable if and only if $G$ (the dual graph?) is Eulerian

    %from ind comb
    \item Let $G$ be a graph.  Then $1 \leq X(G) \leq v$; $X(G) = v$ if and only if $G$ is a complete graph.
    \item cor if $G$ has a subgraph isomorphic to the complete graph $K_p$, then $p \leq X(G)$.
    \item Let $G$ be a graph with at least one edge.  Then $X(G)=2$ if and only if $G$ is a bipartite graph.
    \item 

\end{itemize}
