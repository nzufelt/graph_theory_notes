\section{Colorings}\label{sec:colorings}

\begin{itemize}
    \item A graph has been ($n$-)\textbf{colored} if each vertex has been assigned a number from $\{1, 2, \ldots, n\}$ such that no edge joins vertices with the same number (``color'').  We say that a graph $G$ is \textbf{$n$-colorable} if it may be $n$-colored.
    \item examples
    \item The \textbf{chromatic number} of a graph $G$ is the smallest $n$ such that $G$ is $n$-colorable, denoted $X(G)$.
    \item examples
    \item (DNP) Four-color theorem: Every planar graph has $X \leq 4$.
    \item Five-color theorem: Every planar graph has $X \leq 5$. (induction)
    \item Every planar graph having a vertex of degree $\leq 4$ has $X\leq 4$.  This is crazy!  Compare to the theorem about every planar graph having a vertex $\leq 5$.
    \item reading about the four-color theorem and its proof.  Do you believe it?
    \item Map colorings!  define dual graph
    \item exercises
\end{itemize}
