\section{Colorings}\label{sec:colorings}

\begin{definition} A graph $G$ is said to have been ($n$-)\textbf{colored} if each vertex has been assigned a (non-unique) number from $\{1, 2, \ldots, n\}$ such that no edge joins vertices with the same number (``color'').  We say that a graph $G$ is \textbf{$n$-colorable} if it may be $n$-colored.
\end{definition}

\begin{definition} The \textbf{chromatic number} of a graph $G$ is the smallest $n$ such that $G$ is $n$-colorable, denoted $X(G)$.
\end{definition}

\begin{examples}\leavevmode
\begin{enumerate}
    \item Determine $X(K_4)$, $X(K_5)$, $X(W_5)$.  Prove that your answer is correct.
    \item Determine $X(K_{m,n})$ for all natural numbers $m$ and $n$. Prove that your answer is correct.
\end{enumerate}
\end{examples}

\begin{exercise} Research the following theorem, and tell a story about its proof.  Do you ``believe'' it?
\end{exercise}

\begin{theorem}[Four-color Theorem, Do Not Prove] Every planar graph has $X \leq 4$.
\end{theorem}

\begin{theorem}[Five-color Theorem] Every planar graph has $X \leq 5$.
\end{theorem}

\begin{remark} The following claim is true, because the Four Color Theorem has been established.  However, that it may not be easy to prove.  Do a bit of research into it; if it seems to give you some ``foothold'' toward proving it, then attempt a proof.
\end{remark}

\begin{claim} Every planar graph having a vertex of degree $\leq 4$ has $X\leq 4$.
\end{claim}

\begin{theorem} It is sufficient to prove the four-color theorem for trivalent graphs, or graphs which are regular of degree three.
\end{theorem}

% TODO!!!!
\begin{exercise} Often the Four Color Theorem is stated in terms of \textbf{maps}.  Define a map, the \textbf{dual graph} of a planar diagram, and how the more ``standard'' statement of the Four Color Theorem relates to ours.
\end{exercise}

\begin{theorem}[Brooks] Let $G$ be a connected graph.  If $G$ is not complete and its largest vertex degree is $n$, then $G$ is $n$-colorable.
\end{theorem}

\begin{examples}  Construct examples of graphs $G$ where Brooks theorem is useful and informative toward determining $X(G)$, and graphs for which Brooks theorem is less informative.
\end{examples}

\begin{claim} A map is 2-colorable if and only if the dual graph is Eulerian.
\end{claim}

\begin{theorem} Let $G$ be a graph.  Then $1 \leq X(G) \leq v$; $X(G) = v$ if and only if $G$ is a complete graph.
\end{theorem}

\begin{corollary} If $G$ has a subgraph isomorphic to the complete graph $K_p$, then $p \leq X(G)$.
\end{corollary}

\begin{theorem} Let $G$ be a graph with at least one edge.  Then $X(G)=2$ if and only if $G$ is a bipartite graph.
\end{theorem}
