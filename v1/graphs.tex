\section{Graphs}\label{sec:graphs}

This is the section on graphs, still to be completed.

\begin{definition}
\begin{enumerate}
    \item \label{def:graph} A \textbf{graph} $G=(V, E)$ is a pair of sets $V$ and $E$, where $V$ is a non-empty set and $E$ is a (possibly empty) set consisting only of two-element sets of the form $\{a, b\}$, where $a \in V$ and $b \in V$.  The set $V=V(G)$ is called the set of \textbf{vertices} of $G$ and the set $E=E(G)$ is called the set of \textbf{edges} of $G$.
    \item The number of vertices in a graph is denoted by $v$ and the number of edges in a graph is denoted by $e$.  It is possible that $v=\infty$ or $e=\infty$, meaning that there is no such (finite) number.
    \item If $e=(v_1, v_2)\in E$, then we say that $e$ \textbf{connects} $v_1$ and $v_2$ and that $v_1$ and $v_2$ are \textbf{adjacent}.
    \item \label{def:graph_diagram} Let $D$ be a subset of a space (typically the Euclidean Plane, $\mathbb{R}^2$) consisting of points and arcs connecting those points, such that the arcs only meet the points in their boundaries.  Given a graph $G$, $D$ is said to be a \textbf{graph diagram} for $G$ if:
    \begin{enumerate}
        \item the vertices of $G$ are in one-to-one correspondance with the points of $D$, and
        \item the edges of $G$ are in one-to-one correspondance with the arcs of $D$.
    Note that a graph diagram $D$ is sometimes referred to as an \textbf{embedding}, particularly if the space is not $\mathbb{R}^2$.
    \end{enumerate}

    \item Two graphs are \textbf{equal} if they have equal vertex and edge sets.  Two graph diagrams are equal if they represent equal graphs.
\end{enumerate}
\end{definition}

Another name for vertex is \textit{node}.

\begin{lemma} Let $G$ be a graph.  Then $G$ has no \textnormal{loops, i.e.} edges connecting a vertex to itself, and $G$ has no \textnormal{skeins, i.e.} collections of more than one edge connecting a pair of vertices.
\end{lemma}

{\color{red} directed vs. undirected}

\begin{example}
    \begin{enumerate}
        \item null graph
        \item cyclic graph $C_v$
        \item complete graph $K_v$
    \end{enumerate}
\end{example}

\begin{theorem} \label{thm:edges_in_Kv} Let $K_v$ denote the complete graph on $v$ vertices.  Then $e=|E(K_v)|= \frac{1}{2} v \cdot (v-1)$.
\end{theorem}

\begin{definition} compliment and subgraph, isomorphism, supergraph
\end{definition}

equal implies isomorphic

isomorphism is an equivalence relation

isomorphism implies equal numbers of Vs and Es

definition of degree/valence
% Is ``even'' and ``odd'' vertex defined somewhere?  I need it for euler

isomorphism implies the set of degrees is the same and the number of vertices of degree $n$ is the same.

non-isomorphism examples, e.g. \verb|non-iso_graphs.png| \label{ex:non-iso_graphs}

\section*{exercises (interject above)}
\begin{enumerate}
    \item wheel graphs, draw some and prove number of edges.
    \item determine and prove the number of edges in $\overline{G}$ given data about $G$.
    \item determine all numbers $v$ such that $C_v \cong K_v$. prove.
    \item Prove that $C_v \cong \overline{C_v}$ if and only if $v = 5$.
    \item $G \cong \overline{G}$ implies that $v$ or $v-1$ is divisible by 4.
    \item Maybe theorem: if $G_1 \cong G_2$ and $A_1 \subseteq G_1$ then there exists a subgraph $A_2 \subseteq G_2$ with $A_1 \cong A_2$.  Then, reprove the non-iso from \ref{ex:non-iso_graphs}
    \item ? prove the number of isomorphism classes of v=3 is 7.  ``Classify all graphs with 3 vertices up to isomorphism.''
    \item $G_1 \cong G_2$ iff $\overline{G_1} \cong \overline{G_2}$.
    \item define bipartite, prove non-isomorphism of a pair
        \item Let $X$ and $Y$ be a partition of the vertices....
    \item Create formulae for the number of edges of each of the following families of graphs, based on the number of vertices ($v$):
    \begin{itemize}
        \item $K_v$
        \item $C_v$
        \item $\overline{C}_v$
        \item $W_v$ (wheel graph)
        \item $K_{m,n}$.
    \end{itemize}

    % from int. comb.
    \item degree sequence e.g. (5,4,3,3,3,3,3,2,2) as an isom invt, find a pair with at least 4 verts for which the invt fails
    \item draw a connected graph with degree sequence equal to that above.
    \item Prove that a graph with $v \geq 2$ has two vertices with the same degree.
    \item Prove that any two connected graphs with the same number of vertices and degree sequences $(2, 2, \ldots, 2)$ are isomorphic.
    \item Prove that a graph with at least
    $$\frac{(n-1)(n-2)}{2} +1$$
    edges must be connected.  Find an example to prove that this bound is ``sharp'', \textit{i.e.} that this bound cannot be improved in general.
    \item

\end{enumerate}
