\section{Planar Graphs} \label{sec:planar}
\begin{definition}\leavevmode
\begin{enumerate}
    \item In a graph diagram $D$ in the space $X$ (such as the plane, $\mathbb{R}^2$), an \textbf{edge-crossing} is a point $x\in X$ such that $x$ lies in the interior of more than one edge/arc of $D$.
    \item A graph $G$ is \textbf{planar} if it has a diagram drawn in the plane with no edge-crossings.
\end{enumerate}
\end{definition}

\begin{examples}\leavevmode
\begin{enumerate}
    \item Draw a diagram of $K_4$ which has an edge-crossing, and one without.  Is $K_4$ planar?
    \item Draw a planar diagram for $K_{m,n}$ for $m<3$ and $n<4$.
\end{enumerate}
\end{examples}

\begin{theorem}[Jordan Curve Theorem, Do Not Prove] If $C$ is a continuous simple closed curve in a plane and two points $x$ and $y$ of $C$ are joined by a continuous simple arc $L$ such that $L \cap C = \{x, y\}$, then except for its endpoints $L$ is entirely contained in one of the two regions of $\R^2 \setminus C$.
\end{theorem}

\begin{exercise} Draw an example of the situation described in the Jordan Curve theorem which demonstrates that the conclusion of the Jordan Curve Theorem is not always obvious.
\end{exercise}

\begin{theorem}\leavevmode
\begin{enumerate}
    \item $K_{3,3}$ is not planar.
    \item $K_5$ is not planar.
\end{enumerate}
(Do not use Theorem \ref{thm:euler}).
\end{theorem}

\begin{theorem} Any subgraph of a planar graph is planar.
\end{theorem}

\begin{corollary} Any supergraph of a nonplanar graph is nonplanar.
\end{corollary}

\begin{definition} Let $G$ and $H$ be graphs.
\begin{enumerate}
    \item If $G$ may be obtained from $H$ by replacing an edge $(x, y)$ of $H$ with another vertex $v$, and a pair of edges $(x, v), (v, y)$, then $G$ is said to be obtained from $H$ via an \textbf{edge expansion}.
    \item If $G$ may be obtained from $H$ by a finite sequence of edge expansions, then $G$ is an \textbf{expansion} of $H$.
    \item If $G$ is obtained from $H$ by a finite sequence of expansions and passing to supergraphs, then $G$ is said to be an \textbf{expanded supergraph} of $H$.
\end{enumerate}
\end{definition}

\begin{theorem} Let $G$ and $H$ be graphs. If $G$ is an expanded supergraph of $H$, then $G$ is a supergraph of an expansion of $H$.
\end{theorem}

\begin{theorem} Every expanded supergraph of $K_{3,3}$ or $K_5$ is nonplanar.
\end{theorem}

\begin{theorem}[Kuratowski's Theorem, Do Not Prove] A graph $G$ is nonplanar if and only if $G$ is an expanded supergraph of $K_{3,3}$ or $K_5$.
\end{theorem}

\begin{example} Construct an example of a ``large'' graph which is indeed planar by Kuratowski's Theorem.
\end{example}

\subsection{Euler's Formula}
\begin{definition} Given a planar graph diagram $D$ of a graph $G$, a \textbf{face} of $D$ is the set of all points in $\R^2 \setminus D$ that may be joined by a continuous arc in $\R^2 \setminus D$.  The number of faces of $D$ is denoted as $f(D)$ (or just $f$ or $f(G)$ if $D$ is clear from context.)
\end{definition}

\begin{theorem} If $G$ is a planar graph with planar diagrams $D_1$ and $D_2$, then $f(D_1) = f(D_2)$.
\end{theorem}

\begin{definition} A graph diagram $D$ is \textbf{polygonal} if it is planar, connected, and has the propery that every edge of $D$ borders on two distinct faces.
\end{definition}

\begin{theorem}[Euler]\label{thm:euler} If $G$ is polygonal then $v-e+f = 2$.
\end{theorem}

\begin{corollary} If $G$ is planar and connected, then $v-e+f = 2$.
\end{corollary}

\begin{corollary} $K_5$ and $K_{3,3}$ are nonplanar.
\end{corollary}

\begin{theorem} If $G$ is planar then $G$ has a vertex of degree $\leq 5$
\end{theorem}

\begin{definition} Let $d\in \mathbb{N}$.  A graph is \textbf{regular of degree} $d$ if all its vertices have degree $d$.
\end{definition}

\begin{examples} Construct an example of a connected graph with $v$ vertices which is regular of degree $d$ for $v = 5, 6,$ and $7$ and $d=2, 3,$ and $4$, if possible.  If it isn't possible, explain why.
\end{examples}

\begin{definition} A graph is \textbf{platonic} if it is polygonal, regular, and all its faces are bounded by the same number of edges.
\end{definition}

\begin{examples}\leavevmode
\begin{enumerate}
    \item Construct an example of a platonic graph which is regular of degree $4$.
    \item Construct an example of a polygonal, regular graph which is not platonic.
\end{enumerate}
\end{examples}

\noindent We now aim to classify the platonic graphs.  In order to do so, we'll need the following lemmas.

\begin{lemma} If $G$ is regular of degree $d$ then $e=\frac{dv}{2}$.
\end{lemma}

\begin{lemma} If $G$ is platonic of degree $d$, and $n$ is the number of edges bounding each face, then $f = dv/n$.
\end{lemma}

\begin{theorem}[Euclid] Apart from $K_1$ and the cyclic graphs, there are 5 platonic graphs.
\end{theorem}
