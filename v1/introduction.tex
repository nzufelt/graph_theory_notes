\section{Introduction}\label{sec:introduction}

%%%%% Useful reminders %%%%%
% \cite{pap:gabaifoliations3}

% \subsection{Construction of the graphs $G_Q$ and $G_P$ in the three summands case}

% \ref{thm:threesummandsbound}

%\begin{theorem}\label{thm:threesummandsbound} If Dehn surgery of slope $r$ on a knot $K$ in $S^3$ produces a manifold with more than two, and hence three connected summands, then $|r|\leq b$, where $b$ is the bridge number of $K$. Consequently, the Two Summands Conjecture holds for knots with $b\leq 5$.
%\end{theorem}

% \begin{corollary}\label{cor:2sc_positive_braids} Let $K$ be the closure of a positive braid in $S^3$.  Then Dehn surgery on $K$ produces at most two prime connected summands.
% \end{corollary}

% \begin{question}
% For which classes of knots is there a relationship between the bridge number of a knot and the difference $e-n$ of a minimal-strand braid representing the knot?
% \end{question}

% \begin{definition} Let $\Lambda$ be a great web in the either the general or the three summands case.  Then $\Lambda$ is \textnormal{small} if there exists a collection of $\frac{p}{2}$ parallel edges in $\Lambda$ containing no Scharlemann cycles, otherwise $\Lambda$ is \textnormal{large}.  Such a collection of parallel edges is called a \textnormal{full quota}.
% \end{definition}

% \begin{proposition}  \label{prop:largewebs}Let $\Lambda$ be a great web in either case.  Then $\Lambda$ is large.
% \end{proposition}

% \begin{proof}[Proof of Corollary \ref{cor:primeweb}]
% It is an immediate consequence of \cite[Lemma 2.4]{pap:howiethreesummands} that

% \end{proof}

% \begin{figure}[hbt]
% \begin{center}
% \includegraphics[width = .4\textwidth]{pionerelations}
% \caption{The disks extending $B_i$ to give the relation $g_{a+i}g_{a-i}=1$.}
% \label{fig:pionerelations}
% \end{center}
% \end{figure}

% \begin{figure}[hbt]
% \begin{center}
% \begin{subfigure}{.49\textwidth}
% \begin{center}
% \includegraphics[width = .75\textwidth]{findingbandmeridian}
% \caption{$K'$ is a band sum in $S^3_r(K)$, and $J$ is a meridian of the band $B$.}
% \label{fig:findingbandmeridian}
% \end{center}
% \end{subfigure}
% \begin{subfigure}{.49\textwidth}
% \begin{center}
% \includegraphics[width = .75\textwidth]{unknotting.eps}
% \caption{Passing pieces of the band and the knot $K'$ over the disk bounded by $J'$ (not pictured).}
% \label{fig:unknotting}
% \end{center}
% \end{subfigure}
% \caption{}
% \end{center}
% \end{figure}

% \begin{align*}
% k_1 + (k_1+1)\cdot(nl_2-1) &= k_2 + (k_2+1)\cdot(nl_1-1),\\
% k_1 + nl_2 + k_1nl_2-k_1 -1 &= k_2 + nl_1 +k_2nl_1-k_2-1,\\
% k_1l_2 &= k_2l_1.
% \end{align*}


%%%%% begin content %%%%%
Topics to cover:
Introduction: who am I, what is this course
what is a Proof
what is graph theory
what are the topics we need to cover
what depends on what


% To add an unnumbered section to the ToC, use:
% \addcontentsline{toc}{section}{Acknowledgments}
\section*{Acknowledgments}
The author would like to thank ...
