\section{Trees}\label{sec:trees}

\begin{definition} A \textbf{tree} is a connected graph with no circuits.
\end{definition}

\begin{theorem} Let $T$ be a graph.  The following are equivalent:
\begin{enumerate}
    \item $T$ is a tree,
    \item $T$ contains no circuits/cycles and has $v-1$ edges,
    \item $T$ is connected and has $v-1$ edges,
    \item $T$ is connected and every edge of $T$ is a \textbf{bridge}, \textit{i.e.} it lies in a disconnecting set of a size one,
    \item Any two vertices of $T$ have a unique path conntecting them,
    \item $T$ contains no circuits/cycles but the addition of any new edge produces exactly one circuit/cycle.
\end{enumerate}
\end{theorem}

\begin{theorem} Let $T$ be a tree.  If $T$ has a vertex of degree $p$, then it has at least $p$ \textit{leaves}, \textit{i.e.} vertices of degree $1$.
\end{theorem}

\begin{exercise}Determine the number of non-isomorphic trees of size:
\begin{enumerate}
    \item 5
    \item 6
    \item 7
\end{enumerate}
\end{exercise}

\begin{definition} Let $G$ be a connected graph.  A \textbf{spanning tree} for $G$ is a subgraph $T$ of $G$ that has the properties:
\begin{enumerate}
    \item $T$ is a tree, and
    \item every vertex of $G$ is a vertex of $T$.
\end{enumerate}
\end{definition}

\begin{exercise} Construct an algorithm to ``grow'' a spanning tree, \textit{i.e.} given an edge $e$ of a connected graph $G$, describe an algorithm which will result in a spanning tree containing $e$.  Prove that your algorithm always works.
\end{exercise}

\begin{exercise} What can be said (and proven) about those edges that are contained in every spanning tree?  In none?
\end{exercise}

\begin{theorem} Let $G$ be a graph with $v$ vertices.  Then the length of the shortest circuit that includes each edge of $G$ at least once is $2(v-1)$.
\end{theorem}

\begin{exercise} The following theorem is due to Kirchhoff, and is quite challenging.  Do not attempt to prove it!  Look up what all the terms mean, be able to describe it to a classmate, compute a couple of examples and perhaps even look up a proof of the theorem!
\end{exercise}

\begin{theorem}[Kirchhoff's Theorem] Let $G$ be a connected graph.  Construct the following matrix $L=L(G)$, called the \textit{Laplacian matrix} of $G$.  Order the vertices of $G$ as $v_1, v_2, \ldots, v_v$, and then $L_{i,j}$, the entry of $L$ located in entry $i,j$, is given by:
\begin{equation*}
L_{i,j} = \left\{
\begin{array}{ll}
\text{deg}(v_i)& \text{if }i=j\\
-1 & \text{if $i \neq j$ and $i$ and $j$ are adjacent}, or\\
0& \text{otherwise.}
\end{array}
\right.
\end{equation*}
Then $L$ is a singular matrix, and so has determinant 0.  However, any submatrix of rank $n-1$, \textit{i.e.} any matrix obtained from $L$ by deleting the $i^{\text{th}}$ row and column, for some $i=1, \ldots, v$, is not singular, and the absolute value of its determinant is equal to the number of distinct spanning trees of $G$.
\end{theorem}

\begin{corollary}[Cayley's Theorem] For a complete graph $K_n$, the number of spanning trees of $K_n$ is $n^{n-2}$.
\end{corollary}
