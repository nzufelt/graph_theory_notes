\section{Set Theory}\label{sec:settheory}

\begin{definition}
\begin{enumerate}
    \item \label{def:set} A \textbf{Set} is a collection of distinct objects, none of which is the set itself.  If $a$ is an object lying in the set $A$, we write $a in A$.
    \item \label{def:nullset} A set containing no elements is called the \textbf{empty set}, or the \textbf{null set}, and is written $\emptyset$ or $\{ \}$.
    \item \label{def:subset} A set $A$ is said to be a \textbf{subset} of the set $B$, written $A \subseteq B$ if every element of $A$ is also an element of $B$.
    \item \label{def:setequality} A set $A$ is said to be a \textbf{equal to} the set $B$, written $A = B$ if $A \subseteq B$ and $B \subseteq A$.
\end{enumerate}
\end{definition}

If it is possible to enumerate the elements of $A$, we do so with the notation:
$$ A = \{ a, \pi, \frac{45}{36}, ``Massachusetts''\}.$$

Add some discussion about what this means?

\begin{theorem}\label{thm:nullsetunique} There is only one empty set.
\end{theorem}

\begin{lemma}\label{lem:subsettransitivity} (transitivity of subset) If $A \subseteq B$ and $B \subseteq C$, then $A \subseteq C$.
\end{lemma}

Other possibilities: sets are subsets of themselves, sets are not elements of themselves.

cardinality

bijection/one-to-one correspondence


\section*{exercises (interject above)}

list subsets of $\{1, 2, 3\}$.

how many subsets of a given set? prove!

the natural numbers are in bijection with the even numbers
